\chapter{Noter}

\section{Data}
\textcolor{red}{Alle csv-filerne sættes sammen i en liste(vektor), så hvert element indeholder en tabel over målingerne for én bygning.}

\textcolor{red}{Start- og sluttid for målingerne laves om til dato-format.}

\textcolor{red}{X-kolonnen fjernes, da den kun består af NAs.}


\section{Exploratory Analysis}
\textcolor{red}{pairs plot for hver bygning}

\textcolor{red}{Helt generelt kan vi se, at flowet generelt er lavere om sommeren.}

\textcolor{red}{Smartest at lave et nyt datasæt for vejrdata, hvor vi flipper sættet, så det seneste datopunkt skal være 29. januar kl. 07:00:00.}

\textcolor{red}{Q-plot for at undersøge hvad sammenhængen er mellem energiforbruget (consumption) og udendørstemperaturen. Tilføjer en abline til hvor vi vil sige der slukkes for varmen i huset. Der laves noget least squares på data på hver side af abline, og så fokuserer vi jo self på det der ligger i den kolde periode. Dog kan vi undersøge hvordan huset performer ved at kigge på data i månederne uden varme.}