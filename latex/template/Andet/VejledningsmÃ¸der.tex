\chapter{Vejledningsmøder}

\section{19. februar}

\subsection{Spørgsmål}
\begin{enumerate}
    \item Hvorfor er der nogle af husene, som kun har omkring 3600 observationer, mens andre har 9400? Hvad vil det betyde for os? Hvad kan vi gøre? \textcolor{red}{Vi skal i sidste ende lave noget der virker på tilgængeligt data. Realistisk problem hæhæ. Vi må godt sige, at vi skal have nok data. En delopgave: hvor mange data skal der til for at kunne sige noget konstruktivt. Ændrer det på konklusionerne?Få denne perspektivering ind på et eller andet tidspunkt.}
    \item Må vi fjerne hus 5? Den giver os problemer... \textcolor{red}{Vi skal bare ændre på datoerne for hus 5 inde i en text editor.}
\end{enumerate}

\subsection{Noter}
\begin{itemize}
    \item Hvornår er der informationer nok, hvornår er der ikke?
    \item Når vi laver vores modeller, skal vi lave dem således at mængden at data kan variere. Man laver noget for hvert hus, så man så kan sammenligne et eller andet. Hvad er ens, og hvad er forskelligt for hvert hus? 
    \item Lasse forventer ikke, at vi ender med perfekte modeller. Thank God! 
    \item Brug \texttt{as.POSIX} til at lave tiden. Kig på input- og outputtype. 
    \item Der er to måder at lave varmt brugsvarme på - enten varmeveksler eller varmevandsbeholder. Beholder: hvis temp. i bunden bliver for lav - opvarmningen bliver dermed mere jævn. Pladevarmeveksler: ligesom radiator, fjernvarme igennem radiatoren og brugsvarme i midten eller sådan noget. 
    \item Vi har også sommerdata - kig på varmeforbruget der til at få en idé om hvordan huset opfører sig. Er der et hårdt forbrug mellem kl. 7-8? Maj eller september måned kan vise hvordan deres varmevandsforbrug er. Er der peaks, eller er det jævnt fordelt? 
    \item Man skal ikke kaste for meget væk.
    \item Brugsvand er støj, men det ikke tilfældig støj. Det er positivt, så det påvirker estimaterne. Noget af det kan vi fjerne, men vi skal se på data hvor der ikke er varme - er der nogle mønstre? 
    \item Hvilken ugedag er bedst til at repræsentere en weekend? Måske lørdage? 
    \item Skal vi kigge på hvordan huset performer, eller skal vi kigge på hvordan huset performer her og nu? 
    \item Hvor stopper vi? Det vigtigste er, at vi laver nogle ting, som vi ved kommer til at virke.
    \item Teoridelen: det er vigtigere at vi får tydeliggjort hvad den her metode kan.  
\end{itemize}

\subsection{Hvad skal vi?}
\begin{itemize}
    \item Tjek forskel på ugedage, weekender, helligdage, ferier - hvad gør vi med disse forskelle? 
    \item Få lavet plots. 
    \item Markér underlig opførsel i data i plots.
    \item Find de normale perioder og så gør noget dér. Alt det andet kigger vi på senere. 
\end{itemize}

\pagebreak

\section{26. februar}

\lesstodo[inline]{Daily averages of consumption versus temperature differences}

\todo[inline]{Læse artikler fra Peder}

\subsection{Spørgsmål}
\begin{enumerate}
    \item abline på Q-plot - kan vi optimere den på nogen måde, eller er det okay vi bare vælger en temperatur? Det er meget realistisk, at folk tænder for varmen, når der er under 13 grader udenfor. Vi har brug for en smart måde at optimere på. \textcolor{red}{Vi kan sagtens optimere denne. Vi skal dog lave plottet på døgnværdier i stedet.}
    \item Hvordan sorterer man rækkerne i et data.frame ud fra en bestemt søjle? \textcolor{red}{Den her er vist fikset.}
    \item Idéen var at udfylde de punkter vi mangler og så fylde dem ud med NA værdier. Så rækkerne mangler ikke, men de er tomme. Er det en korrekt måde at håndtere dette problem på? \textcolor{red}{Peder siger det giver mening og så tage højde for det derfra. Det giver mening fordi det er samplet meget skarpt. Lav en vektor med de tidspunkter vi gerne vil have og så merge data.frame med vektoren og så keep left, så fylder den ind. Husk én detalje: sommertid og vintertid.}
    \item Vise plots - er det godt eller skidt? 
\end{enumerate} 

\subsection{Noter}
\begin{itemize}
    \item Al data er højst sandsynligt målt i samme tidszone. 
    \item Peders strategi: fortæl den at det er "GMT" eller "UTC" tid.
    \item Vi laver en model for hvert hus, fordi det skalerer til mange huse. 69 forskellige sæt parametre men det kan godt være samme model. Det er en af de diskussioner vi kommer til at skulle lave.  
    \item Hvad effekten af at bruge forskellige modeller? Der kommer forskellige ting ind, vi kan sammenligne huse, hvor mange data har man? Hvilken betydning har det? 
    \item Vi tager ét hus - hvad kan vi gøre med en månedsdata og så laver vi et rullende vindue. Hvilke estimater et eller andet. Er det faktisk robust det vi har gang i? Plot parameter estimaterne gør nok noget henover året. Hvad gør konfidensintervallerne? 
    \item Brug subset af data til at estimere med, forskellige længder, overlap osv. Det er en god måde at lave robuste modeller på. Kan man fx overhovedet se at folks juleferier har betydning? 
    \item I første omgang er det at kigge på hvordan husene opfører sig. Vi starter med at bygge ting op, som vi ved virker. Forudsigelse og undersøgelse af robusthed. 
    \item Tag en eller to dages gennemsnit på varmesæsonen og så tage parametrene og plot dem for den model eller så noget. 
    \item Normaliseret pr. kvadratmeter i huset. 
    \item Når vi ikke har indetemperaturen, er vi nødt til at have mu med. Hvis man bruger en masse el, så påvirker det også estimatet af indetemperaturen. 
    \item Plot af hele data, pairs plot, vinterperioder - plot for alle sammen. Fx et hus der opfører sig helt gakket. 
    \item Det plot med knækket vi har - vi skal tage det over hele dagen og ikke baseret på timerne. Man kan også lave en model, hvor man tager autokorrelationen med og så bruger weighted least squares. 
    \item \texttt{aggregate} fra Peder.
    \item Hvis man laver modelreduktion - hvad er altid med? Brug \texttt{step}-funktionen til at reducere. Er weekdays signifikant? 
    \item Helsingørdata: Nogenlunde samme modeller som for Aalborg. Vi har el og vand og vil lave dagsværdier, hvad kan vi bruge det til? Hvad hvis vi ikke bruger el og vand, hvad hvis vi gør? Får vi merværdi. 
\end{itemize}

\subsection{Hvad skal vi lave?}
\begin{itemize} 
    \item Lave vektor og merge med data.frame
    \item Lave projektplan: kursusbeskrivelse og læringsmål ligesom for et kursus. Brug teksten fra mda'en eller sådan noget. 10 linjer eller noget. Hvad er læringsmål, som vi skal måles på? 
    \item Hvad er egentlig det nye vi laver/undersøger?
\end{itemize}

\pagebreak

\section{5. marts}
\lesstodo[inline]{få styr på lorte parskip-pakken}
\todo[inline]{Få aksefis af Grønning eller Maika}

\subsection{Spørgsmål}
\begin{itemize}
    \item Vi vil gerne aflevere den 20. juni, så vi kan fremlægge senest den 27. juni. 
    \item Hvad er det helt præcist volume er? Umiddelbart ville vi mene det var det samme som flow, men værdierne er forskellige og flow er pr. time mens volume ikke er.
    \item Vil det have nogen betydning senere hen, hvis vi har fjernet EndDateTime nu?
    \item Hvad skal vi lægge i korrelationerne? Fortæl os det.
\end{itemize}

\subsection{Hvad skal vi have lavet?}
\begin{itemize}
    \item Læse notefis grundigt. 
    \item Kigge på fejl i optim-funktion (Anton).
    \item Få styr på ggplot. 
\end{itemize}

\subsection{Noter}
\subsubsection{Til projektplan:}
\begin{itemize}
    \item SEAS-NVE vil gerne vide hvad for nogle forskellige ting man kan lave med de data. 
    \item Sammenligne huse - hvad kan vi sammenligne, hvad kan vi ikke sammenligne? 
    \item Hvad er det de godt vil kommunikere til beboerne på den lange bane? Hvor godt performer beboerens hus. 
    \item Relativt sammenlignelige huse - hvordan er deres temperaturafhængighed? 
    \item Hvad der er signifikant ligger bagved.
    \item Forecasts: hvad er der af døgnvariationer? Er der specifikke mønstre? Der er nogen der har en brændeovn - kan vi se om den er tændt? varmevandsforbrug - hvordan er det fordelt på døgnet? Har man natsænkning/dagssænkning? 
    \item Til tidsrækkedelen: det skal være en dynamisk model. Der er en overførselsfunktion, der kan være svær at identificere. 
    \item Døgnvariation som ikke kobler til dynamikken og heller ikke temperaturen, DET er det spændende, siger Lasse!
    \item Kør to ting parallelt, når vi er tre. 
    \item IKKE START MED ET HUS MED BRÆNDEOVN!
    \item Vi skal nok lege manuelt med et par forskellige huse og så tage den derfra. 
    \item Fix punkt 2 og så kommentarer omkring hvordan det skal kommunikeres.
\end{itemize}

\subsubsection{Andet:}
\begin{itemize}
    \item Optimeringen af $\alpha$ ligger udenfor - i optim.
    \item Se Lasses tegning - lav en funktion som hedder piecewise
    \item Hvis der er huller i data: lave rå-gennemsnit og så bagefter se hvornår et eller andet.
    \item Kig på residualer fra en model. Lav plot på dagsværdier og håb på ting ser mere robust ud. 
    \item Varians inhomogenitet????? Plot residualerne mod de forskellige variable. 
    \item Variablen flow er det flow der er her og nu, når målingen laves. Volumen er det flow der er løbet igennem siden sidste måling. Lasse forventer, at volumen er det robuste tal. 
    \item Flow og temperature kommer fra EndDateTime, så det er det vi skal bruge. 
    \item Til Anders: Noget med volumen og de temperaturer vi har her er de fra øjeblikket eller er de for den forgående time. 
    \item Energidata: kan vi gange volumen og temp.forskellen sammen og få noget der ligner energidata. FØR VI SKRIVER TIL ANDERS. 
    \item Energi er vist ikke electricity consumption. 
    \item Vi skal sige til Anders, at alle huses data ser sådan her ud. Men inden skal vi lige tjekke forholdet mellem volumen og coolingdegree. 
    \item Energi burde være 4.186 blabla * temp forskel * flow
    \item Kig på dagsværdier nu. 
\end{itemize}

\pagebreak

\section{12. marts}
Bestemmelse af max temp. Hvor stabil er hældningen? Jo flere data der er med i modellen, jo bedre er estimatet. Men når de dårlige
værdier inkluderes bliver det dårligere igen.
Kig på diagnostic plot af de fittede huse. 
Lav en linear model med backward selection. Fuld regressionsmodel. Se på hvilke variable som er vigtige. Sammenlign hældninger.
Kig på hvad de forskellige variable gør. Hvilke variable skal med i modellen og hvilke skal bruges til at estimere parametre.
Man tager tit varmeforbrug pr. kvadratmeter. Enten ved at dividere forbruget eller hved at scalere paranmetre.
Når vi snakker tidsrække modeller skal vi se på ACF.

\pagebreak

\section{19. marts}

\subsection{Spørgsmål}
\begin{itemize}
    \item Fortælle om vores bud på at bestemme overgangsperioden for fjernvarme.
\end{itemize}

\subsection{Noter}
\begin{itemize}
    \item Lav heat maps til exploratory
    \item I forhold til at bestemme $\alpha$, så se billede af Mikkels figur. Når den går over for good i standard afvigelser, så er det det punkt et eller andet. De har valgt 3. Lav noget qsum eller sådan noget. Når standard afvigelserne skal plottes skal de transformeres med log +1. Bruger +20 grader som træningsdata. Vælger et robust estimat af middelværdi og standard afvigelse. 
    \item Denne måde går oftest rigtig godt, men det kan gå dårligt, hvis det er 5 grader, og de så skruer ned for varmen. Også offentlige bygninger med weekendsænkning. 
    \item Vælge et andet sigma niveau og så gøre det i 1-grads intervaller og se gvor mange der ligger under. Finn og Anton har styr på det. 
    \item Der er en prior på 13 grader (det er for at være robuste), og hvis vi så har mere information, så gør vi noget andet. 
    \item Det en fordeling fra 10 til 16 grader, hvor de fleste ligger på 13-14 grader. 
    \item Lave et kriterie (threshold) for hvor mange observationer der skal ligge i 20+.  
    \item Man burde måske bruge første kvartil for de huse hvor vi ikke har nok data, da man begår færre fejl ved at vælge en for lav værdi. 
    \item Hvornår ved man at man kan stole på metoden? Lasse plejer at sige 12 grader. 
    \item Lave et afsnit om hvordan man finder breakpointet - hvilket kapitel? 
    \item Man får et større estimat af variansen, når der er ferier, weekend osv. Det gode ville være at lave en multiple linear model og tilføje parametre som påvirker forbruget et eller andet. Hvor mange af husene har sådanne effekter, og hvornår har de ikke de effekter. 
    \item http://skoleferie-dk.dk/skoleferie-aalborg/
    \item Dag til dag variationen er mindre, fordi en af energikilderne forsvinder. 
    \item Vi laver en lineær regressionsmodel, og vi kigger på variansen som funktion af periode. Hvis modellen er stort set perfekt til at forudsige, når der ikke bliver brugt varmt vand, så bør man pille disse perioder ud af modellen. De skal ikke smides ud i første omgang, men farv dem og se om de ligger anderledes end de andre. Tjek perioderne signifikans først. 
    \item Kig på autokorrelationen i lag 1 for nogle huse. Er der væsentlige korrelation - ja eller nej? Sikre sig at man kigger på de rigtige lags. Det er stadig dagsværdier. Skal vi korregere for det? Lave weighted least squares i stedet for bare least squares.
    \item cuesum senere hen. 
\end{itemize}

\subsection{Hvad skal vi lave?}
\begin{itemize}
    \item Lav et plot for alle autokorrelationer for husene. 
    \item Få lavet en standard måde vi plotter data for alle husene på. Split husene op alt efter hvor mange observationer der er. 
\end{itemize}

\pagebreak

\section{26. marts}

\subsection{Noter}
\begin{itemize}
    \item På time niveau når du går længere ind i appen, har de en simpel døgnkurve model. Noget med en døgnmodel med simple gennemsnit. 
    \item akkumuleret, cumulativt plot, hvor man summerer op i energi. På dagsværdier gør det ikke så stor forskel. 
    \item Kig på hvor mange af de interpolerede værdier der rent faktisk er. Hvis der er mere end 2 decimaler, så er det interpoleret. Bare lav modulus. 
    \item Hvordan er hældningerne i forhold til areal, bygningsår, hvilken type hus. 
    \item Vi kan sagtens smide temp. variabel ind i ggplots. Loess laver en trekant.
    \item Smid temperaturen ind også, fordi det er den vi forventer der har mest indflydelse. Lav den kun som funktion af temperaturen og så plot residualer over datoerne. 
    \item Hvis man skal være pragmatisk, så skal vi vælge et fast tal. For at få variansen ned skal vi holder os skarpt under 15 og over 10. 12-13 grader. 
    \item Det der kendetegner at der er varme i forhold til brugsvarme, er at der er flow hele vejen, så vi skal kigge på flow. 
    \item Hvordan kan vi blande solen ind? Sunhour og condition. Sunhour kan kun have en værdi forskellig fra nul, når condition er 0,1,2. 
    \item \textbf{Multiple linear regression model:} 

\end{itemize}

\subsection{Hvad skal vi lave?}
\begin{itemize}
    \item LAV SIMPLE LINEAR REGRESSION MODELS
\end{itemize}

\pagebreak

\section{5. april}

\subsection{Spørgsmål}
\begin{itemize}
    \item Hvordan skal vi tolke det, når temperaturen ikke er signifikant?
    \item Skal vi forsøge at trække vores parametre ud af vores modeller?
    \item Vise vores model - er det rigtigt?
    \item Skal vi overveje at transformere vores data, når nu vi har vores QQ plots? 
\end{itemize}

\subsection{Noter}