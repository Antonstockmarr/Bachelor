\chapter{Vejledningsmøder}

\section{19. februar}

\subsection{Spørgsmål}
\begin{enumerate}
    \item Hvorfor er der nogle af husene, som kun har omkring 3600 observationer, mens andre har 9400? Hvad vil det betyde for os? Hvad kan vi gøre? \textcolor{red}{Vi skal i sidste ende lave noget der virker på tilgængeligt data. Realistisk problem hæhæ. Vi må godt sige, at vi skal have nok data. En delopgave: hvor mange data skal der til for at kunne sige noget konstruktivt. Ændrer det på konklusionerne?Få denne perspektivering ind på et eller andet tidspunkt.}
    \item Må vi fjerne hus 5? Den giver os problemer... \textcolor{red}{Vi skal bare ændre på datoerne for hus 5 inde i en text editor.}
\end{enumerate}

\subsection{Noter}
\begin{itemize}
    \item Hvornår er der informationer nok, hvornår er der ikke?
    \item Når vi laver vores modeller, skal vi lave dem således at mængden at data kan variere. Man laver noget for hvert hus, så man så kan sammenligne et eller andet. Hvad er ens, og hvad er forskelligt for hvert hus? 
    \item Lasse forventer ikke, at vi ender med perfekte modeller. Thank God! 
    \item Brug \texttt{as.POSIX} til at lave tiden. Kig på input- og outputtype. 
    \item Der er to måder at lave varmt brugsvarme på - enten varmeveksler eller varmevandsbeholder. Beholder: hvis temp. i bunden bliver for lav - opvarmningen bliver dermed mere jævn. Pladevarmeveksler: ligesom radiator, fjernvarme igennem radiatoren og brugsvarme i midten eller sådan noget. 
    \item Vi har også sommerdata - kig på varmeforbruget der til at få en idé om hvordan huset opfører sig. Er der et hårdt forbrug mellem kl. 7-8? Maj eller september måned kan vise hvordan deres varmevandsforbrug er. Er der peaks, eller er det jævnt fordelt? 
    \item Man skal ikke kaste for meget væk.
    \item Brugsvand er støj, men det ikke tilfældig støj. Det er positivt, så det påvirker estimaterne. Noget af det kan vi fjerne, men vi skal se på data hvor der ikke er varme - er der nogle mønstre? 
    \item Hvilken ugedag er bedst til at repræsentere en weekend? Måske lørdage? 
    \item Skal vi kigge på hvordan huset performer, eller skal vi kigge på hvordan huset performer her og nu? 
    \item Hvor stopper vi? Det vigtigste er, at vi laver nogle ting, som vi ved kommer til at virke.
    \item Teoridelen: det er vigtigere at vi får tydeliggjort hvad den her metode kan.  
\end{itemize}

\subsection{Hvad skal vi?}
\begin{itemize}
    \item Tjek forskel på ugedage, weekender, helligdage, ferier - hvad gør vi med disse forskelle? 
    \item Få lavet plots. 
    \item Markér underlig opførsel i data i plots.
    \item Find de normale perioder og så gør noget dér. Alt det andet kigger vi på senere. 
\end{itemize}

\pagebreak

\section{26. februar}

\lesstodo[inline]{Daily averages of consumption versus temperature differences}

\todo[inline]{Læse artikler fra Peder}

\subsection{Spørgsmål}
\begin{enumerate}
    \item abline på Q-plot - kan vi optimere den på nogen måde, eller er det okay vi bare vælger en temperatur? Det er meget realistisk, at folk tænder for varmen, når der er under 13 grader udenfor. Vi har brug for en smart måde at optimere på. \textcolor{red}{Vi kan sagtens optimere denne. Vi skal dog lave plottet på døgnværdier i stedet.}
    \item Hvordan sorterer man rækkerne i et data.frame ud fra en bestemt søjle? \textcolor{red}{Den her er vist fikset.}
    \item Idéen var at udfylde de punkter vi mangler og så fylde dem ud med NA værdier. Så rækkerne mangler ikke, men de er tomme. Er det en korrekt måde at håndtere dette problem på? \textcolor{red}{Peder siger det giver mening og så tage højde for det derfra. Det giver mening fordi det er samplet meget skarpt. Lav en vektor med de tidspunkter vi gerne vil have og så merge data.frame med vektoren og så keep left, så fylder den ind. Husk én detalje: sommertid og vintertid.}
    \item Vise plots - er det godt eller skidt? 
\end{enumerate} 

\subsection{Noter}
\begin{itemize}
    \item Al data er højst sandsynligt målt i samme tidszone. 
    \item Peders strategi: fortæl den at det er "GMT" eller "UTC" tid.
    \item Vi laver en model for hvert hus, fordi det skalerer til mange huse. 69 forskellige sæt parametre men det kan godt være samme model. Det er en af de diskussioner vi kommer til at skulle lave.  
    \item Hvad effekten af at bruge forskellige modeller? Der kommer forskellige ting ind, vi kan sammenligne huse, hvor mange data har man? Hvilken betydning har det? 
    \item Vi tager ét hus - hvad kan vi gøre med en månedsdata og så laver vi et rullende vindue. Hvilke estimater et eller andet. Er det faktisk robust det vi har gang i? Plot parameter estimaterne gør nok noget henover året. Hvad gør konfidensintervallerne? 
    \item Brug subset af data til at estimere med, forskellige længder, overlap osv. Det er en god måde at lave robuste modeller på. Kan man fx overhovedet se at folks juleferier har betydning? 
    \item I første omgang er det at kigge på hvordan husene opfører sig. Vi starter med at bygge ting op, som vi ved virker. Forudsigelse og undersøgelse af robusthed. 
    \item Tag en eller to dages gennemsnit på varmesæsonen og så tage parametrene og plot dem for den model eller så noget. 
    \item Normaliseret pr. kvadratmeter i huset. 
    \item Når vi ikke har indetemperaturen, er vi nødt til at have mu med. Hvis man bruger en masse el, så påvirker det også estimatet af indetemperaturen. 
    \item Plot af hele data, pairs plot, vinterperioder - plot for alle sammen. Fx et hus der opfører sig helt gakket. 
    \item Det plot med knækket vi har - vi skal tage det over hele dagen og ikke baseret på timerne. Man kan også lave en model, hvor man tager autokorrelationen med og så bruger weighted least squares. 
    \item \texttt{aggregate} fra Peder.
    \item Hvis man laver modelreduktion - hvad er altid med? Brug \texttt{step}-funktionen til at reducere. Er weekdays signifikant? 
    \item Helsingørdata: Nogenlunde samme modeller som for Aalborg. Vi har el og vand og vil lave dagsværdier, hvad kan vi bruge det til? Hvad hvis vi ikke bruger el og vand, hvad hvis vi gør? Får vi merværdi. 
\end{itemize}

\subsection{Hvad skal vi lave?}
\begin{itemize} 
    \item Lave vektor og merge med data.frame
    \item Lave projektplan: kursusbeskrivelse og læringsmål ligesom for et kursus. Brug teksten fra mda'en eller sådan noget. 10 linjer eller noget. Hvad er læringsmål, som vi skal måles på? 
    \item Hvad er egentlig det nye vi laver/undersøger?
\end{itemize}

\pagebreak

\section{5. marts}

\subsection{Hvad skal vi have lavet?}

