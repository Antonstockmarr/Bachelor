%!TEX root = ../Thesis.tex
\chapter{Introduction}
\textcolor{red}{Not done} \\

\noindent According to the International Energy Agency \cite{iea}, heat is the largest energy end-use. Providing heating for homes and industrial purposes accounts for around 50\% of the total energy consumption. Renewable heat consumption in the form of bioenergy contribution is expected to grow which will be a better solution for the climate. In relation to the individual consumer, it makes sense to become aware of one's heat consumption, e.g many consumers pay more for their heat consumption than they could. This can be solved by making small adjustments such as replacing radiators with more efficient cooling, improve insulation of the house etc. Which factors that can influence the heat consumption, are not known to most consumers and thus it can be a challenge to know how to minimize the consumption. \\

\noindent Heat consumption can be described using mathematical models, namely statistical models, and this can lead to an optimization/minimization of the consumption.


\section{Motivation}
The aim of this report is to investigate the tap water consumption and thereby provide possible extensions to the app created by SEAS-NVE, WATTS. By illustrating these features in the app, customers can become aware of their heat consumption and at the same time get a sense of what physical phenomena affect their house. \\

\noindent Our approach is to develop statistical models in order to analyse which factors influence the heat consumption.

\section{Introduction to WATTS app}
The app WATTS is designed and created by the danish energy and optical fibre broadband concern. 