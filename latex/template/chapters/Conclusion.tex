%!TEX root = ../Thesis.tex
\chapter{Conclusion}
In this project statistical models have been developed to fit the consumption data for a number of houses, provided by SEAS-NVE. The goal was to generate estimates that could describe the physical factors that have an influence the consumption. The estimates should be as precise and robust as possible, and they should be visualised in a way that makes them easy for a consumer to interpret. They should be aimed at being included in WATTS.

\noindent Two different models were applied to the data, a linear regression model and a time series model. The linear regression model was used on the daily consumption values. The assumptions of the model were supported by most of the houses, and it used the outside temperature, the solar radiation and the wind as input variables. The model provided estimates that had a reasonable precision, and the same model could easily be applied to new houses and give similar results. The estimates of this model were used to compare the houses and to illustrate how they were expected to perform in different situations with varying temperature, solar radiation, wind speed and wind direction.

\noindent The time series model was used on the hourly consumption with the hopes of obtaining estimates with better precision than the regression model. Unfortunately, the hourly data was too unstabile to provide meaningful results. Even for the most well behaved house and the model with the best performance, the precision was still not as good as for the regression model. The main reason is expected to be the way that the data is collected. Imprecise time stamps and data interpolation introduces variation in the data that is hard to model. Before further work is put into modelling the data on an hourly basis, this issues should be adressed.

\noindent There are still many ways to improve the regression models. More work can be put into identifying tap water consumption, improving the distinguishing between summer and winter periods, investigation the number of observations required to get proper results, etc. There are also many possible ways to present the data to the consumers that have not yet been explored, e.g. making it possible to compare oneself with houses that are similar in terms of consumption, year of construction or other attributes.

\textcolor{blue}{Den helt overordnede konklusion er, at vi kan lave statistiske modeller til analyse af varmeforbruget på dagsværdier. Men hvis vi gerne vil kigge på det eksakte data, dvs. timeværdierne, så er der desværre for mange usikkerheder ved målingerne \textcolor{red}{noget om hvordan målerne har målt det} og det bliver for svært at analysere varmeforbruget. Vi har forsøgt os med tidsrækkemodeller, men de kan ikke opfange den overordnede svingende tendens, der er i data, hvilket gør dem mere upålidelige. Tidsregistreringen skal være ordentlig, så der rent faktisk er en måling pr. time}
