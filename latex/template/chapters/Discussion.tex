\chapter{Discussion}
\label{chap: discussion}

\begin{itemize}
    \item Data-checking function: den kan udvides, så modellerne bliver mere robuste. Svært at predicte længere frem, hvis der ikke er særlig meget data. Baseret på vores resultater for vores modeller, kan vi godt fjerne nogle huse.
    \item Problemet med predictions er at man må antage vejrudsigterne holder.
    \item Kan den lineære regressionsmodellen bruges? Ved forudsigelser er der store usikkerheder, men for de fleste huse er modellen rigtig god. 
    \item Nogle huse opfører sig yderst mærkværdigt, hvad kan dette skyldes? Hvad foregår der fx i hus 18?
    \item Mangel på informationer om fx hvor mange der bor i husstanden, husets placering \begin{itemize}
        \item[-] tap water, påvirke usikkerhederne for modellerne
        \item[-] vindens påvirkning
        \item[-] husets placering påvirker estimaterne for vinden  
    \end{itemize}
    \item Timeværdierne opfører sig underligt, hvilket formentlig skyldes den måde de er målt på. 
\end{itemize}

\section{Future work}
\begin{itemize}
    \item Google earth og så placere vindafhængighedsplots udenom for at kunne vægte retningerne ordentligt. 
    \item Clustering og se på outliers
    \item Fysisk timemodel 
\end{itemize}