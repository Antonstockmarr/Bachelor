\chapter{Discussion}
\label{chap: discussion}
Now two different types of models have been developed - a linear regression model for describing the heat consumption as daily values and a time series model for the hourly consumption. \\

\noindent For the daily values both, a simple and a multiple regression model have been tried. The simple model only included the temperature as the independent variable. In relation to validation of the model, the diagnotic plots were not met at all. The model did not take several factors into account, which was seen in the residuals behaviour, thus the model was too simple. The multiple model started out as a full model that included all the relevant weather attributes. The purpose was to determine which factors should be included in a general multiple model for comparison of houses. The parameters that were significant for most of the houses were the temperature, solar radiation, the wind directions east and west. In the general regression model, all wind directions and their interaction with temperature were included. The behaviour of the residuals were definitely improved compared to the simple model and the model was concluded to be valid. By investigating the estimates of the temperature from the general model it was seen that the temperature were highly significant with negative sign for all houses and none of the 95\% confidence intervals were above zero. Furthermore, predictions for January 2019 using the model were used to support that the estimates indeed were good. The model was good at capturing the overall trend, and therefore the estimates could be concluded to be good enough to describe the influence of the temperature on the heat consumption. However, the predictions were made under the assumption that the weather forecasts were exact for the chosen period. This causes uncertainties for the predictions as they become better than actual data. Overall, the general regression model performs as it should, and it can be used to describe the daily heat consumption despite the uncertainties of using the model for predictions. \textcolor{red}{Mangler lidt.} \\

\noindent For the hourly values an ARIMA and a MARIMA model have been developed. \textcolor{red}{Bro hjælp.} \\

\noindent As mentioned in Chapter \ref{chap: data} a \textcolor{magenta}{Mickey Mouse} data-checking function has been proposed. The function checks if there is a certain number of observations for each house. One can discuss whether the function could also be used after the model development. Based on the results given in \textcolor{red}{blabla} there are several houses that could probably be omitted because they have too few observations or there are too many interpolations of missing observations in the house measurements. It can be problematic to predict a certian number of days ahead, e.g. 60 days, if there only exists observations for half a year. Furthermore, if there are too many missing values ​​that have been interpolated with the method mentioned in Chapter \ref{chap: data},then a data-checking function should discard the house, as it will have a negative impact on the further modelling of the heat consumption for that specific house. \\

\noindent The lack of information on e.g the number of residents in the house, the house's location, \textcolor{red}{etc.} affects the uncertainties of the models.


\begin{itemize}
    \item Nogle huse opfører sig yderst mærkværdigt, hvad kan dette skyldes? Hvad foregår der fx i hus 18?
    \item Mangel på informationer om fx hvor mange der bor i husstanden, husets placering \begin{itemize}
        \item[-] tap water, påvirke usikkerhederne for modellerne
        \item[-] vindens påvirkning
        \item[-] husets placering påvirker estimaterne for vinden  
    \end{itemize}
    \item Timeværdierne opfører sig underligt, hvilket formentlig skyldes den måde de er målt på. 
\end{itemize}

\section{Future work}
\begin{itemize}
    \item Google earth og så placere vindafhængighedsplots udenom for at kunne vægte retningerne ordentligt. 
    \item Clustering og se på outliers
    \item Fysisk timemodel 
    \item Kun lave tidsrækkemodeller over vinterperioden uden huller
    \item Udvidelse af data-checking function
\end{itemize}