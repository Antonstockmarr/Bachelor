\chapter{Discussion}
\label{chap: discussion}
Now two different types of models have been developed - a linear regression model for describing the heat consumption as daily values and a time series model for the hourly consumption. \\

\noindent For the daily values, both a simple and a multiple regression model have been tested. The simple model only included the temperature as the independent variable. The validation of the model assumptions was not met at all. The model did not take several factors into account, which was seen in the residuals behaviour. Thus the model was concluded to be too simple. A multiple model was developed by starting out with a full model that included all the relevant weather attributes. The purpose was to determine which factors should be included in a general multiple model for comparison of houses. The parameters that were significant for most of the houses were the temperature, the solar radiation and the wind directions east and west. Based on this, a more general model was developed, including these parameters together with the interaction betweeen each of the wind directions and the temperature. The behaviour of the residuals were definitely improved compared to the simple model and the model was concluded to be valid.

\noindent By applying this general model to the available data, estimates of the temperature were highly significant with negative sign for all houses and none of the 95\% confidence intervals were above zero. These estimates were concluded to have a good reliability. Estimates were also calculated for the solar radiation and the wind directions. The model was also tested by making forecasts for January 2019. The model was good at capturing the overall trend, and therefore the estimates could be concluded to be good enough to describe the influence of the temperature on the heat consumption. The estimates provided by the model can be of use to the developers of WATTS in several ways. The temperature estimates give an indication of how dependent the consumption of a house is on temperature variations. Houses with high temperature estimates are vulnerable to cold weather, and will probably have a high consumption in the winter period. The estimates can be used to compare or group houses together. Alternatively, SEAS-NVE can consider contacting the consumers with high estimates to do troubleshooting.\\

\noindent Additionally, the models can also be used to present the forecasts to the users, giving them an overview of their heat consumption that is very similar to the one implemented in WATTS for the energy consumption. However, one should be aware that the predictions were made under the assumption that the weather forecasts were assumed to be exact for the chosen period. This causes uncertainties for the predictions as they become better than they would for uncertain weather forecasts The wind estimates of the models can also be visualised to the consumer. By illustrating how the house is affected by wind from different directions, they can get an idea of where to look for leaks, bad insulation or other targets for investments. If the location and orientation of a house is known, the wind estimates can be compared to the surroundings of the house. Then one can consider if the neighbours, nearby trees or hills have an effect on the consumption of that house. \\

\noindent In an attempt to improve the temperature estimates found by the linear regression model, different time series models was used on the data. First different models were tested and evaluated using the \texttt{arima} function. A single model was developed that could be applied to every house in the data set. The model was used to make predictions for unknown data, but it was not able to describe the consumption in a very satisfying way because random variations in the hour data were too dominating. A modification of the data was made to remove some of the variations from the data, but the model still did not perform well. Then a series of different models was developed using the \texttt{Marima} package in \texttt{R}. Different models were applied to a handful of houses. The models were used to compute the step response for each of the houses. But even for the best models modified with Fourier series, the estimates of the step response were much more uncertain than the temperature estimates found with the regression models. The validation of the regression model was much better than the validation of the time series model, so this is clearly the estimate that should be trusted. \\

\noindent In summary, the hourly data turned out to be too unstable to apply any time series models to at this point. A lot of time and work can go into evaluating different models, but if the data does not support it, then it will not be of much use. When there is as much unexplained variation in the data as there is now, then the models will desperately try to make a fit, making them just as unreliable. The root of the problem is how the observations are sampled by Aalborg Forsyning. As long as the observations are not initially divided into clear cut one hour intervals, the hourly data will not be consistent enough to base models on. By converting the measurements into daily values, the strange behavior of the measurements becomes less evident, which results in the regression models being used to describe the daily heat consumption. \\

\noindent But the linear regression model is not perfect either. Some houses, like house 18, still have some unpredictable behavior on the daily level, making the model assumptions worse. The tap water consumption is not taken into account in the model, and that introduces a lot of uncertainty. If there was a way to gain more knowledge on the tap water consumption of a house, e.g. if they have a hot-water tank, then the estimates might be improved. The lack of other information on e.g the number of residents in the house, the house's location, etc. also affects the uncertainties of the models. The tap water consumption can depend on how many people live in the household, and there is therefore a high probability that the hot water consumption is more significant when there are more residents per square meter.\\

\noindent As mentioned in Chapter \ref{chap: data} a function has been proposed that can reject houses from being modelled if they do not fulfill certain criteria. The function checks if there is a certain number of observations for each house. One can discuss whether the function could also be used after the model development. Based on the results given in Chapter \ref{chap: daily} there are several houses that could probably be omitted because they have too few observations or there are too many interpolations of missing observations in the house measurements. It can be problematic to predict a certian number of days ahead, e.g. 60 days, if there only exists observations for half a year. Furthermore, if there are too many missing values ​​that have been interpolated with the method mentioned in Chapter \ref{chap: data}, then a data-checking function should discard the house, as it will have a negative impact on the further modelling of the heat consumption for that specific house. \\





\section{Future work}
Although the quality of data on an hourly basis has not been good enough to construct time series models, there is still something to work on in the project. At the same time, there is also a part that can be improved in relation to data and the way data is measured. \\

\noindent As mentioned, the models for the hourly values do not ​​work particularly well, so one of the things one could primarily work on, is to achieve more stable data. It does not make much sense to continue working on the models, as they cannot produce any useful results at this time. Another way of working on the time series models in the future, could be to investigate the winter period. The heat consumption is low during the summer, so it is the winter period that is of interest, and the models could possibly be improved by this action. \\

\noindent In proportion to data on a daily basis, an immediate idea would be to compare the wind dependency plots presented in \cref{fig: WKplot18} and \cref{fig: WKplot55} with the location of the houses and then use Google Earth to visualize the consumers house surrounded by a compas and the coloured shapes. Future work could also be a clustering that compares the houses' consumption with other houses in the same class. The clustering could be made based on the way the houses are heated, the houses' areas, year of construction etc.

\noindent The distribution of the tap water consumption is not quite clear, so if it could be identified when houses have some sort of "type" consumption such as wood stove, water pump, etc.,it would be possible to remove the tap water consumption from data. It would most likely result in an improvement in the performance of the models. Knowledge of the different heating patterns and methods in the houses, could then be used to classify the houses, and specific models could thus be developed for the different classes.

\noindent The method used for choosing the break point explained in Chapter \ref{chap: exploratory} could also be constructed in a more sofisticated way. And last but not least, the Data Checking function could be expanded such that it could determine how much data is needed for one house in order to make the models more robust. For example, one could also consider removing houses where the heat consumption is zero for an extended period, such as house 18.
