\chapter{Discussion}
\label{chap: discussion}
Now two different types of models have been developed - a linear regression model for describing the heat consumption as daily values and a time series model for the hourly consumption. \\

\noindent For the daily values both, a simple and a multiple regression model have been tried. The simple model only included the temperature as the independent variable. In relation to validation of the model, the diagnotic plots were not met at all. The model did not take several factors into account, which was seen in the residuals behaviour, thus the model was too simple. The multiple model started out as a full model that included all the relevant weather attributes. The purpose was to determine which factors should be included in a general multiple model for comparison of houses. The parameters that were significant for most of the houses were the temperature, solar radiation, the wind directions east and west. In the general regression model, all wind directions and their interaction with temperature were included. The behaviour of the residuals were definitely improved compared to the simple model and the model was concluded to be valid. By investigating the estimates of the temperature from the general model it was seen that the temperature were highly significant with negative sign for all houses and none of the 95\% confidence intervals were above zero. Furthermore, predictions for January 2019 using the model were used to support that the estimates indeed were good. The model was good at capturing the overall trend, and therefore the estimates could be concluded to be good enough to describe the influence of the temperature on the heat consumption. However, the predictions were made under the assumption that the weather forecasts were exact for the chosen period. This causes uncertainties for the predictions as they become better than actual data. Overall, the general regression model performs as it should, and it can be used to describe the daily heat consumption despite the uncertainties of using the model for predictions. \textcolor{red}{Mangler lidt.} \\

\noindent For the hourly values an ARIMA and a MARIMA model have been developed. \textcolor{red}{Bro hjælp.} \\

\noindent As mentioned in Chapter \ref{chap: data} a \textcolor{magenta}{Mickey Mouse} data-checking function has been proposed. The function checks if there is a certain number of observations for each house. One can discuss whether the function could also be used after the model development. Based on the results given in \textcolor{red}{blabla} there are several houses that could probably be omitted because they have too few observations or there are too many interpolations of missing observations in the house measurements. It can be problematic to predict a certian number of days ahead, e.g. 60 days, if there only exists observations for half a year. Furthermore, if there are too many missing values ​​that have been interpolated with the method mentioned in Chapter \ref{chap: data},then a data-checking function should discard the house, as it will have a negative impact on the further modelling of the heat consumption for that specific house. \\

\noindent The lack of information on e.g the number of residents in the house, the house's location, etc. affects the uncertainties of the models. The tap water consumption can depend on how many people live in the household, and there is therefore a high probability that the hot water consumption is higher than the heat consumption. This creates greater uncertainty as the models do not take the tap water consumption into account. The location of the house affects the estimates of the four different wind directions. If the location of each house was known, it would be easier to interpret the estimates. For example, if a terrace house is located between two other houses, it would make sense that the heat consumption would be less affected when the wind came from the directions where the house adjoins the other two houses. \\

\noindent Overall, there is a major problem with the way data is collected. It is clearly seen on an hourly basis, since data is measured per hour. By converting the measurements into daily values, the strange behavior of the measurements becomes less evident, which results in the regression models being used to describe the daily heat consumption. The biggest challenge then arises when the time series models are used to describe the hourly consumption.


\begin{itemize}
    \item Nogle huse opfører sig yderst mærkværdigt, hvad kan dette skyldes? Hvad foregår der fx i hus 18?
    \item Timeværdierne opfører sig underligt, hvilket formentlig skyldes den måde de er målt på. 
\end{itemize}

\section{Future work}
Although the quality of data on an hourly basis has not been good enough to construct time series models, there is still something to work on in the project. At the same time, there is also a part that can be improved in relation to data and the way data is measured. \\

\noindent 

Modellen fungerer ikke særlig godt for tidsfuck, så en af de ting man først og fremmest kan arbejde med er at opnå mere stabil data. Det giver ikke ret meget mening at arbejde videre med modellerne, da de ikke kan give noget brugbart på nuværende tidspunkt. Hvis man nu skulle arbejde videre med tidsfuck så kunne man se på at kun bruge på vinterperioden til modelleringen. Så får vi ikke sommerlort, hvor der ikke er så meget heat consumption. Vi er mest intra i vinterperioden hvor der rent faktisk er tændt for varmen, så man kunne muligvis få nogle bedre modeller ved kun at kigge på vinterperioden.

I forhold til daglig data, det mest umiddelbare ville være at tage windplots og sammenholde dem med deres placering og så bruge google earth til visualiering for kundefuck. 

Man kunne også lave clustering, som siger noget om hvor godt et hus klarer sig i forhold til lignende huse. Man kunne lave det efter, hvordan de er varmet op, hvor stort huset er,hvornår huse er bygget. 

Man kunne lave en bedre måde at lave breakpointet på. Hinte til hvordan Grønning og Maika har gjort. De har lavet en udvidelsesalgoritme, hvor de fitter nogle fordelinger. Hvis nogen havde en større viden om det, ville det være en god idé at gøre det.

Data-checking function: arbejde videre med hvor meget data der skal være for ét hus for at modellerne kan blive mere robuste. Man kunne fx overveje at fjerne huse hvor forbruget er 0 over en længere periode. 

Man kunne forsøge at få en bedre idé om hvilken del af forbruget der er tap water. Hvis man kunne identificere hvornår nogle huse har en type forbrug, brændeovn, vandpumpe osv, så kunne man fjerne forbrugsvandet, og så ville modellerne højst sandsynligt performe bedre, da man adskiller tap water fra data. Hvis man inkludere noget viden om de forskellige mønstre i husene, så kunne man classificere husene, og så lave specifikke modeller for hver klasse. 