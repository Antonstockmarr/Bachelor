\chapter{Models on the Hourly Consumption}






\begin{figure}
    \centering
    \includegraphics[width=.8\textwidth]{../../../figures/Heatmap.pdf}
    \caption{}
    \label{fig: daily_cons}
\end{figure}








\section{ARIMAX}
An ARIMAX model is constructed for the hourly data of every house. The outside temperature is used as the exogenous variates. The exploratory analysis has shown that there is a high covariance between the temperature and the consumption, making this the obvious choice. The model is expected to have a seasonal component with season 24, because the consumption in certain time periods are likely to be close to each other from day to day.

In the following a general modelling approach will be used where a large model is applied at first. It is then reduced by looking at the standard deviation of the estimates on all houses. The initial model is an ARIMAX(2,2,2) model with a seasonal (0,1,1) component. This is a very extensive model, and differencing is used to make it stationary. \cref{arimax1_55} shows an example of how the model performs on house $55$, while \cref{arimax1_18} shows the same model for house $18$. For house $18$ both the ACF and the PACF ressemble white noise fairly well, with a few exceptions. The same goes for house $55$. Both have a few significant lags, and they both have a significant lag around lag $61$, even though it is barely so. It is supricing that neither house have lags in the seasons that are even close to being significant. This indicates that the dependencies from day to day might not be as important as it was initially assumed. But as mentioned, this model has some parameters that might not be necessary to include.




\begin{figure}
    \centering
    \includegraphics[width=0.8\textwidth]{../../../figures/arimax/Arimax1_18.jpeg}
    \caption{The autocorrelation function for the ARIMAX(2,2,2)$\times$(0,1,1) model, based on the data from house 18.}
    \label{arimax1_18}
\end{figure}


\begin{figure}
    \centering
    \includegraphics[width=0.8\textwidth]{../../../figures/arimax/Arimax1_55.jpeg}
    \caption{The partial autocorrelation function for the ARIMAX(2,2,2)$\times$(0,1,1) model, based on the data from house 18.}
    \label{arimax1_55}
\end{figure}

\begin{table}[]
    \begin{tabular}{lllllllll}
    Parameters     & AR1  & AR2  & MA1  & MA2  & SMA1 & Intercept & Temperature & Average LogLik \\
    Insignificance & 24\% & 81\% & 15\% & 13\% & 0\%  & 0\%       & 3\%         & -77           
    \end{tabular}
    \end{table}

Der er ikke noget i lag 24. Seasonal er ret ligegyldigt.
ARIMAX er en del bedre end ARIMA med temperatur som covariate.
Vi kører en (1,0,1) model. De fleste ser fine ud. Men nogen har ret signifikante lags, specielt in pacf.
Derfor tilføjes ar(2) og ma(2). Der er mange forskellige modeller og i nogen er bestemte parametrer meget insignifikante.
Der kommer også warnings for nogen af dem.
qq-lines er fine for mange af modellerne, dårlige for nogen af dem.
Nogen huse er helt hen i vejret (Øv, hus 28). De har meget oscillering.
