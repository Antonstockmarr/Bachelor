\chapter{Data}
The data is provided by SEAS-NVE in three data sets. The house data consists of 71 \texttt{.csv}-files containing 8 attributes for each house which is \textcolor{red}{antal} data points in all. The second data set includes weather data containing 11,845 observations with 11 attributes. Furhtermore, the third data set is from Bygnings- og Boligregistret (BBR) and contain details for each of the houses e.g. total area, year of construction and type of house. \textcolor{red}{Mangler muligvis lidt mere her.}

\noindent The main focus of this section will be how data is prepared for the further analysis. 

\section{Original data}
The original house and weather data include hourly observations from the period 31-12-2017 to 29-01-2019. The time period varies in the house data which will be taken into account when cleaning the data. 

Table \ref{tab: housedata} below shows the attributes from the house data set. 
\begin{table}[H]
    \centering
    \begin{tabular}{ll}
     \hline
     \textbf{Variable} & \textbf{Description} \\
    \hline
    \hline
    StartDateTime  &  Start time and date for measurements. Hourly values.\\
    EndDateTime  &  End time and date for measurements.\\
    Energy  &  Electricity consumption in \textit{kWh}.\\
    Flow  &  Amount of water passed through meter in \textit{$m^3/$hour}.\\
    Volume & in $m^3$.\\
    TemperatureIn  &  Temp. of the water flowing into a house in Degrees/C. \\
    TemperatureOut  & Temp. of the water flowing out of a house in Degrees/C.\\
    CoolingDegree  &  Difference between Temp.In and Temp.Out in Degrees/C. \\
    \hline
    \end{tabular}
    \caption{Attributes from the original house data.}
    \label{tab: housedata}
\end{table}

The weather data set consists of the attributes seen in Table \ref{tab: weatherdata}.
\begin{table}[H]
    \centering
    \begin{tabular}{ll}
     \hline
     \textbf{Variable} & \textbf{Description} \\
    \hline
    \hline
    StartDateTime  &  Start time and date for measurements. Hourly values.\\
    Temperature  &  Temperature outside in Degrees/C. \\
    WindSpeed  &  \\
    WindDirection  &  \\
    SunHour  &  \\
    Condition  & \\
    UltravioletIndex  &   \\
    MeanSeaLevelPressure  & \\
    DewPoint  &  \\
    Humidity  &  \\
    PrecipitationProbability & \\
    IsHistoricalEstimated & \\
    \hline
    \end{tabular}
    \caption{Attributes from the original weather data.}
    \label{tab: weatherdata}
\end{table}   

StartDateTime and EndDateTime are always one hour apart. When there 
are missing observations the following the next StartDateTime is simply delayed.
Energy is the measured energy consumption on the meter in the houses.


\section{Cleaning and preparation}
In this section, it is described how the raw data is cleaned and prepared for the statistical analysis. \textcolor{red}{Synes der mangler et eller andet her.}

Both weather data and the house data are aggregated in order to convert hourly values into daily values since there are of interest when modelling i chapter 3. 
\textcolor{red}{Loader en temporary data ind, som vi modificerer indtil vi putter den ind i vores endelige data.} Data from 2017 in the house data are removed since data for the same period is missing in the weather data. 
The format for the attributes \texttt{StartDateTime} and \texttt{EndDateTime} is changed to d-m-Y H:min:sec.
%Vi sætter navnet på den første attribute til StartDateTime. Vi ændrer formatet på de to første attributes til posix, som er $\%d-\%m-\%Y \%H:\%M:\%S$.
Likewise, the attribute \texttt{StartDateTime} in the weather data is converted to the same format as in the house data in order to merge the two data sets. 

For nogle huse er der nogle hourly measurements der ikke er der. Der er huller i målingerne. Disse udfyldes med null, hvilket er bedre/lettere at arbejde med. 

%enddays og startdays sættes for hvert hus - hvornår starter målingerne og hvornår slutter målinger. Tidspunkterne for aller første og aller sidste måling. 

Attributen IsHistoricalEstimated ændres til logical, så vi kan compute med den. 

Vi laver så temp. weather data så vi kan merge det med house data. 
Vi merger ikke al data, da mængden vil være en del større. Vi merger tmp weather data på house data i model processen. 

In the house data there are some measurements missing and it can therefore be difficult to do modelling for the houses in question. To avoid these difficulties, a so called "Data Checking" function has been made in order to check whether several constraints for the data are fulfilled. There most be a certain number of observations and the amount of missing data should not exceed a certain fraction of the data. 


Vi tilføjer en attribute Holiday, som er defineret som working days, christmas break, winter break, spring break, autumn break 

\textcolor{red}{WindDirection og WindSpeed over en hel dag, så vi kan få det til 1 tal.}

\textcolor{red}{Vi tilføjer endnu en attribut til vores weather data som udregner solindstrålingen for en bestemt dag. Dette er en 'bedre' måde at udregne 'solen' på end ved at bruge attributterne SunHour, UltravioletIndex og Condition.}