\chapter{Data}
The data is provided by SEAS-NVE in two data sets. The house data consists of 69 \texttt{.csv}-files containing 8 attributes for each house which is 499,499 data points in all. The second data set includes weather data containing 11,845 observations with 11 attributes. \textcolor{red}{Noget med hvordan data er blevet målt - hvilket udstyr, af hvilken virksomhed osv.} 
The main focus of this section will be how data is prepared for the further analysis.

\section{Original data}
The original house and weather data include hourly observations from the period 31-12-2017 to 29-01-2019. The time period varies in the house data which will be taken into account when cleaning the data. 

Table \ref{tab: housedata} below shows the attributes from the house data set. 
\begin{table}[H]
    \centering
    \begin{tabular}{ll}
     \hline
     \textbf{Variable} & \textbf{Description} \\
    \hline
    \hline
    StartDateTime  &  Start time and date for measurements. Hourly values.\\
    EndDateTime  &  End time and date for measurements.\\
    Energy  &  Electricity consumption in \textit{kWh}.\\
    Flow  &  Amount of water passed through meter in \textit{$m^3/$hour}.\\
    Volume in $m^3$.\\
    TemperatureIn  &  Temp. of the water flowing into a house in Degrees/C. \\
    TemperatureOut  & Temp. of the water flowing out of a house in Degrees/C.\\
    CoolingDegree  &  Difference between Temp.In and Temp.Out in Degrees/C. \\
    \hline
    \end{tabular}
    \caption{Attributes from the original house data.}
    \label{tab: housedata}
\end{table}

The weather data set consists of the attributes seen in Table \ref{tab: weatherdata}.
\begin{table}[H]
    \centering
    \begin{tabular}{ll}
     \hline
     \textbf{Variable} & \textbf{Description} \\
    \hline
    \hline
    StartDateTime  &  Start time and date for measurements. Hourly values.\\
    Temperature  &  Temperature outside in Degrees/C. \\
    WindSpeed  &  \\
    WindDirection  &  \\
    SunHour  &  \\
    Condition  & \\
    UltravioletIndex  &   \\
    MeanSeaLevelPressure  & \\
    DewPoint  &  \\
    Humidity  &  \\
    PrecipitationProbability & \\
    IsHistoricalEstimated & \\
    \hline
    \end{tabular}
    \caption{Attributes from the original weather data.}
    \label{tab: weatherdata}
\end{table}   


\section{Cleaning and preparation}
In this section, it is described how the raw data is cleaned and prepared for the statistical analysis. 

Data er aggregated for at lave timeværdierne om til dagsværdier

Loader en temperary data ind, som vi modificerer indtil vi putter den ind i vores endelige data. 
Vi sætter navnet på den første attribute til StartDateTime. Vi ændrer formatet på de to første attributes til posix, som er $\%d-\%m-\%Y \%H:\%M:\%S$.

Så fjerner vi data fra 2017, fordi vi ikke har noget weather data der. 21 observationer. 

For nogle huse er der nogle hourly measurements der ikke er der. Der er huller i målingerne. Disse udfyldes med null, hvilket er bedre/lettere at arbejde med. 

enddays og startdays sættes for hvert hus - hvornår starter målingerne og hvornår slutter målinger. Tidspunkterne for aller første og aller sidste måling. 

StartDateTime i weather formateres til rette format, så det passer med house data. 

Attributen IsHistoricalEstimated ændres til logical, så vi kan compute med den. 

Vi laver så temp. weather data så vi kan merge det med house data. 
Vi merger ikke al data, da mængden vil være en del større. Vi merger tmp weather data på house data i model processen. 

% plot1 <- ggplot(data = dt.full, mapping = aes(day, EnergyPurchase)) + geom_point() +
%ggtitle("Average consumption for all houses during a year ") + xlab("Time")
%+ ylab("Average consumption (kwh)")
%+ geom_smooth(col=col.plot, se = T)

