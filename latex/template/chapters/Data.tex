\chapter{Data}
The data is provided by SEAS-NVE in three data sets. The house data consists of 71 \texttt{.csv}-files containing 8 attributes for each house which is 513877 data points in all. The second data set includes weather data containing 10,140 observations and predictions of the next 2283 data points, all with 11 attributes. Furthermore, the third data set is from Bygnings- og Boligregistret (BBR) and contain details for each of the houses e.g. total area, year of construction and type of house. 
The main focus of this section will be how this data is prepared for the further analysis. 

\section{Original data}
The original house and weather data include hourly observations from the period 31-12-2017 23:00 to 7-02-2019 10:00. The time period varies in the house data which will be taken into account when cleaning the data. 

\noindent Table \ref{tab: housedata} below shows the attributes from the house data set. 
\begin{table}[H]
    \centering
    \begin{tabular}{ll}
     \hline
     \textbf{Variable} & \textbf{Description} \\
    \hline
    \hline
    StartDateTime  &  Start time and date for measurements.\\
    EndDateTime  &  End time and date for measurements.\\
    Energy  &  Electricity consumption in \textit{kWh}.\\
    Flow  &  Amount of water passed through meter in \textit{$m^3/$hour}.\\
    Volume & in $m^3$.\\
    TemperatureIn  &  Temp. of the water flowing into a house in Degrees/C. \\
    TemperatureOut  & Temp. of the water flowing out of a house in Degrees/C.\\
    CoolingDegree  &  Difference between Temp.In and Temp.Out in Degrees/C. \\
    \hline
    \end{tabular}
    \caption{Attributes from the original house data.}
    \label{tab: housedata}
\end{table}
The Heat Consumption is defined as 
\begin{equation}
    Q = V\cdot \Delta T
    \label{eq: Q_heat}
\end{equation}

\noindent The weather data set consists of the attributes seen in Table \ref{tab: weatherdata}.
\begin{table}[H]
    \centering
    \begin{tabular}{ll}
     \hline
     \textbf{Variable} & \textbf{Description} \\
    \hline
    \hline
    StartDateTime  &  Start time and date for measurements. Hourly values.\\
    Temperature  &  Temperature outside in Degrees/C. \\
    WindSpeed  & Wind speed in $m/s$\\
    WindDirection  & Wind direction i degrees from 0 to 360, 0 being North \\
    SunHour  & The level of sunshine in the hour in a scale from 0 to 1 \\
    Condition  & The weather condition given in numbers between \cite{condition} \\
    UltravioletIndex  & The UV index level \\
    MeanSeaLevelPressure  &  \\
    DewPoint  &  \\
    Humidity  &  \\
    PrecipitationProbability & \\
    IsHistoricalEstimated & Binary variable, true if the datapoint is a prediction \\
    \hline
    \end{tabular}
    \caption{Attributes from the original weather data.}
    \label{tab: weatherdata}
\end{table}   

\noindent The BBR data set consists of the attributes seen in Table \ref{tab: BBR}.
\begin{table}[H]
    \centering
    \begin{tabular}{ll}
     \hline
     \textbf{Variable} & \textbf{Description} \\
    \hline
    \hline
    Key  &  The house ID key\\
    HouseType  &  Type of house: Apartment, house, industrial etc. \\
    TotalArea  & The total area of the house in $m^2$ \\
    Floors  & The number of floors in the house \\
    Basement  & How many $m^2$ basement there is in the house \\
    Attic  & How many $m^2$ attic there is in the house \\
    ConstructionYear  & The year of construction for the house  \\
    Surfaces  & The material on the surface of the outdoor walls of the house \\
    ReconstructionYear  & The year of the latest reconstruction of the house \\
    AdditionalHeating  & If there are any additional heating installed in the house. \\ & Fireplace etc. \\
    \hline
    \end{tabular}
    \caption{Attributes from the BBR data.}
    \label{tab: BBR}
\end{table}   


%\textcolor{blue}{StartDateTime and EndDateTime are always one hour apart. When there are missing observations the following the next StartDateTime is simply delayed. Energy is the measured energy consumption on the meter in the houses.}

%\noindent \textcolor{red}{Noget med at vi også har BBR data.}

\section{Cleaning and preparation}
In this section, it is described how the raw data is cleaned and prepared for the statistical analysis. 

\noindent Due to the fact, that \texttt{StartDateTime} and \texttt{EndDateTime} is always one hour apart, it is redundant to use both of the attributes. The obervations of most of the attributes are made at time \texttt{EndDateTime}, and for that reason it is used as \texttt{ObsTime} for the observations. For the weather data set, the observations is made at time \texttt{StartDateTime}, and there is no \texttt{EndDateTime} for this data set. When merging these data sets, \texttt{ObsTime} is alligned with \texttt{StartDateTime}. The format of these attributes is changed to a \texttt{Posixct} value with d-m-Y H:min:sec as the structure.

\noindent Every now and then, one or more data points in a row are missing. When this happen, a data point with NA-values for all of the attributes except \texttt{ObsTime}, is placed in the data set, which makes the data set easier to use in the modelling process. In the data sets there are no indication of whether or not it is weekend. This attribute is added as well as the school holidays.

\noindent Both weather data and the house data are aggregated with mean values for each day in order to convert hourly values into daily values since there are of interest when modelling in chapter 3, two of the attributes is aggregated in a different way, which is explained later.

%\textcolor{red}{Loader en temporary data ind, som vi modificerer indtil vi putter den ind i vores endelige data.}

%enddays og startdays sættes for hvert hus - hvornår starter målingerne og hvornår slutter målinger. Tidspunkterne for aller første og aller sidste måling. 

%Vi laver så temp. weather data så vi kan merge det med house data. 
%Vi merger ikke al data, da mængden vil være en del større. Vi merger tmp weather data på house data i model processen. \textcolor{blue}{Gør vi? Vi laver weatherCons i data.R, så det for daily data er dette i hvert fald ikke sandt :)}

\textcolor{blue}{In the house data there are some measurements missing and it can therefore be difficult to do modelling for the houses in question. To avoid these difficulties, a so called "Data Checking" function has been made in order to check whether several constraints for the data are fulfilled. There must be a certain number of observations and the amount of missing data should not exceed a certain fraction of the data observation period. }

\subsection{Missing values}

\textcolor{red}{Der hvor der mangler data har vi udregnet dem ved interpolation som givet i det der paper fra Anders. For at kunne lave tidsrække modeller, må der ikke være NA's i data. Det var kun 5-6 huller i alt.}


%\noindent \textcolor{red}{Vi tilføjer en binær attribute for hver ferie, og endnu en for weekender. De forskellige ferier vi tager med er christmas break, winter break, spring break, autumn break} \\



\subsection{The sun and the wind}
\noindent A physical factor that could possibly affect the heat consumption is the sun. In raw data, the attributes \texttt{Condition}, \texttt{SunHour},  and \texttt{UltraVioletIndex} can be seen as explanatory variables for the sun. Instead, an attribute, \texttt{Radiation}, is added to calculate the solar radiation for a given day. This attribute is determined with use of the \texttt{R} function \texttt{calcSol} from the library \texttt{solaR}. The ultraviolet index is a measurement of the strength of ultraviolet radiation and since the attribute \texttt{Radiation} is more exact, \texttt{UltravioletIndex} is removed from the weather data set. \\

\noindent Another physical factor that might be of importance is the wind. There are data available for both the wind direction in degrees and the wind speed. When the data is aggregated into daily values, it is important to pay special attention to the wind attributes, since it is not logical to take the average of degree values. For example, the average wind direction of $359$ degrees and $0$ degrees is not $179.5$ degrees. Instead the wind direction and wind speed are interpreted as polar coordinates in a coordinate system. They are converted to rectangular coordinates. Then they are aggregated from hourly values into daily values, and returned to polar coordinates.  When the wind is aggregated this way, wind directions with high wind speeds are weighted higher than wind directions with low wind speeds. Also the problem with the periodicity of the wind direction is solved.


\subsection{Data checking}
\textcolor{blue}{En funktion hvor der er sat nogle krav for hvad der absolut skal være opfyldt for at det giver mening at modellere.}