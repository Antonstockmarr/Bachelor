%!TEX root = ../Thesis.tex
\chapter{Resumé}
Smart meters er måleinstrumenter som bruges til at måle energi-, varme-, og vandforbruget på timebasis for et hus. Når der er foretaget målinger over en længere periode, kan den tilgængelige data bruges til at etablere forskellige statistiske modeller. \\

\noindent I dette projekt er data, målt ved hjælp af smart meters, brugt til at sammenlige varme- og vandforbruget for et antal huse sammen med tilsvarende vejrdata fra den samme periode. Lineære modeller er blevet anvendt på det daglige gennemsnit af aflæsningerne. Modellerne er blevet brugt til at bestemme effekten fra fysiske fænomener på varmeforbruget med fokus på effekten fra udendørstemperaturen, solindstrålingen og vinden. Estimaterne for disse faktorer er blevet udregnet, og de viste sig at være ret pålidelige. 

\noindent Derudover er tidsrækkemodeller blevet udviklet til at analysere målingerne på timebasis målt med smart meters. Step-responsen af temperaturen og solindstrålingen er blevet udregnet og sammenlignet med estimaterne for regressionsmodellen. Desværre understøttede præcisionen af målingerne ikke modellerne og step-responsen kunne ikke forbedre præcisionen af estimaterne fra regressionsmodellen. \\

\noindent Alle resultaterne i dette projekt er rettet mod at blive inkluderet i appen udviklet af SEAS-NVE, kaldet WATTS. Forskellige foreslag til hvordan resultaterne kan visualiseres, er blevet fremsat undervejs i projektet. 