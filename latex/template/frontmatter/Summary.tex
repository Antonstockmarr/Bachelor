%!TEX root = ../Thesis.tex
\chapter{Summary}
Smart meters are pieces of hardware that can measure the energy,the heat and the water consumption of a house on an hourly basis. When readings have been made for an extended period, the available data can be used to establish various statistical models.\\

\noindent In this project the data from the smart meters have been used to compare the heat and the water consumption of a number of houses with weather data from the same period. Linear models have been applied to daily averages of the readings. They have been used to determine the effect of physical phenomena on the consumption, in particular the effect of the outside temperature, the solar radiation and the wind. Estimates for these have been calculated, and they proved to be fairly reliable.

\noindent In addition, time series models have been developed to analyse the smart meter readings on an hourly basis. The step responses of the temperature and the solar radiation were calculated and compared to the estimates of the regression model. Unfortunately, the precision of the readings did not support the models, and the step responses could not improve the precision of the regression model estimates. \\

\noindent All the results in this project were aimed at being included in the app developed by SEAS-NVE called WATTS. Ways of visualising the results have been proposed throughout this paper.